\lstset{
	language	= c++,
	rulesepcolor= \color{gray},
	breaklines=true,
	commentstyle = \color{gray},
	keywordstyle={
		\color[RGB]{175,0,175}
		\bfseries
	},
	basicstyle={
		\footnotesize
		\ttfamily
	},
	tabsize	 = 8,
	showtabs	= false,
	frame	   = single,
	columns	 = fullflexible,
	showstringspaces = false,
}

% Custom lstlisting style
\lstdefinestyle{style}{
	rulesepcolor= \color{gray},
	breaklines=true,
	commentstyle = \color{gray},
	keywordstyle={
		\color[RGB]{175,0,175}
		\bfseries
	},
	basicstyle={
		\tiny
		\ttfamily
	},
	frame	   = single,
	showstringspaces = false,
	extendedchars = true,
	keepspaces = true,
	columns	 = flexible,
	tabsize	 = 8,
}

\newcommand{\song}[1]{{\CJKfamily{song}#1}} % 定义一个命令,用于楷体引用
\newcommand{\kai}[1]{{\CJKfamily{kaiti}#1}} % 定义一个命令,用于楷体引用
\newcommand{\hei}[1]{{\CJKfamily{hei}#1}} % 定义一个命令,用于楷体引用

\newcommand{\blue}[1]{\hei{\color{blue}#1}}
\newcommand{\red}[1]{\hei{\color{red}#1}}

% 自定义一个新的frame命令,自动使用当前最深级别的标题
\usepackage{etoolbox}
\newcommand{\currentdeepesttitle}{}
% 新命令来确定当前最深的标题层级
\newcommand{\determineCurrentTitle}{%
    \ifnumequal{\value{subsubsection}}{0}{%
        \ifnumequal{\value{subsection}}{0}{%
            \renewcommand{\currentdeepesttitle}{\insertsectionhead}%
        }{%
            \renewcommand{\currentdeepesttitle}{\insertsubsectionhead}%
        }%
    }{%
        \renewcommand{\currentdeepesttitle}{\insertsubsubsectionhead}%
    }%
}

% 在每个subsection和subsubsection开始时调用\determineCurrentTitle
\AtBeginSection[]{%
    \determineCurrentTitle%
}
\AtBeginSubsection[]{%
    \determineCurrentTitle%
}
\AtBeginSubsubsection[]{%
    \determineCurrentTitle%
}
